{\bf \Large
\begin{tabular}{ccc}
\hline
  Corresponding author & : & Yuta Kawai\\
\hline
\end{tabular}
}

\section{Coordinate system}
%%
To describe the governing equations for both regional and global dynamical cores, 
a non-orthogonal curvilinear horizontal coordinate $(\xi,\eta)$ 
and a general vertical coordinate $\zeta$ are introduced, 
following \cite{li2020development}. 
For the horizontal coordinate transformation, 
the contravariant form of the metric tensor is represented by $G_h^{ij}$ for $i, j=1, 2$. 
%%
A three-dimensional metric tensor with the horizontal coordinate transformation is defined as 
%%
\begin{align}
    G^{ij} = \begin{pmatrix}
      G_h^{11} & G_h^{12} & 0 \\
      G_h^{21} & G_h^{22} & 0 \\  
      0 & 0 & 1
    \end{pmatrix}. 
\end{align}
%%
The horizontal Jacobian is defined as $\sqrt{G_h}=|G_{h}^{ij}|^{-\frac{1}{2}}$. 
The Christoffel symbol of the second kind  $\Gamma^i_{ml}$,
which means the spatial variation of the basis vector, is represented as 
%%
\begin{align}
  \Gamma^i_{ml} = \dfrac{1}{2} G^{im} \left( 
    \dfrac{\partial G_{jm}}{\partial x^k} 
  + \dfrac{\partial G_{km}}{\partial x^j} 
  + \dfrac{\partial G_{jk}}{\partial x^m} 
  \right)
\end{align}


For the vertical coordinate transformation, 
the metric tensor is defined as  $G_v^{13}=\partial \zeta/\partial \xi, G_v^{23}=\partial \zeta/\partial \eta $ 
and 
the vertical Jacobian is defined as $\sqrt{G_v}=\partial z/\partial \zeta$.  
The vertical velocity in the transformed vertical coordinate can be written 
using contravariant components of wind vector $(u^\xi, u^\eta, u^\zeta)$ as 
%%
\begin{align}
\widetilde{u^\zeta} \equiv \dfrac{d \zeta}{d t} = \dfrac{1}{\sqrt{G_v}}\left( u^\zeta + \sqrt{G_v} G_v^{13} u^\xi + \sqrt{G_v} G_v^{23} u^\eta \right).  
\end{align}
%%


The final Jacobian composed of horizontal and vertical coordinate transformations can be represented as $\sqrt{G}=\sqrt{G_h}\sqrt{G_v}$. 
Hereafter, to briefly describe the formulations, 
the coordinate variables are sometimes expressed using $(\xi^1,\xi^2,\xi^3)=(\xi,\eta,\zeta)$. 
In addition, 
the Einstein summation notation will be applied 
for repeated indices when representing the geometric relations.


\subsection{Horizontal coordinates used in regional model}
%%
For horizontal coordinates in our regional model, 
the horizontal Cartesian coordinates $(x, y)$ is simply adopted 
although we will introduce map projections in the near future. 
Then, 
%%
\begin{equation}
    G_h^{ij} =  \begin{pmatrix}
        1 & 0 \\
        0 & 0
      \end{pmatrix}, \;\;\;
    \sqrt{G_h}=1, 
\label{eq:metric_tensor_Jacobian_regional_model}
\end{equation}
%%
The Christoffel symbol of the second kind  $\Gamma^i_{ml}$ is represented as 
\begin{align}
  \Gamma^1_{ml} = 0, \;\; \Gamma^2_{ml} = 0, \;\; \Gamma^3_{ml} = 0, 
\label{eq:ChristoffelSymbol2ndKind_regional_model}
\end{align}
%%
where $m,l=1,2,3$. 


The components of angular velocity vector included in the Coriolis terms
are given as
%%
\begin{align}
  \Omega^1 = 0, \;\;
  \Omega^2 = 0, \nonumber \\
  \Omega^3 = f_0 + \beta y.
%\label{eq:angular_vel_vec_cartesian}
\end{align}
%%
Here, 
$f_0=2 \Omega \sin{\theta_0}, \beta= 2\Omega \cos{\theta_0}$ 
where 
$\omega$ is the angular velocity of the planet and $\theta_0$ is the reference latitude. 




\subsection{Horizontal coordinates used in global model}
%%
For horizontal coordinates in our global model, 
we adopt an equiangular gnomonic cubed-sphere projection \citep{sadourny1972FDMQuasiUniformSphereGrids,RONCHI199693} 
to map a cube onto a sphere.  
Compared to a conformal projection \citep{RANCC1996}, 
we prefer this projection to generate more uniform grids in high spatial resolutions, 
although the non-orthogonal basis need to be treated.
In each panel of the cube, 
a local coordinate using the central angles 
$(\alpha, \beta)$ ($\in [-\pi/4,\pi/4]$) was introduced and 
related to the horizontal coordinates $(\xi, \eta)$ by $\xi=\alpha, \eta=\beta$. 
Based on the derivation with the coordinate transformation 
in previous studies \citep[e.g.,][]{Nair2005DGTransport,ullrich2012dynamical,li2020development}, 
the horizontal contravariant metric tensor and the horizontal Jacobian for the equiangular gnomonic cubed-sphere projection can be written as, respectively,  
%%
\begin{equation}
    G_c^{ij} = \dfrac{\delta^2}{r^2(1+X^2)(1+Y^2)} \begin{pmatrix}
        1+Y^2 & XY \\
        XY    & 1 + X^2
      \end{pmatrix}, \;\;\;
    \sqrt{G_c}=\dfrac{r^2 (1+X^2)(1+Y^2)}{\delta^3}, 
\label{eq:metric_tensor_Jacobian_cubedsphere}
\end{equation}
%%
where $X=\tan \alpha$, $Y=\tan \beta$, $\delta=\sqrt{1+X^2+Y^2}$, and $r$ is the radial coordinate.  
The Christoffel symbol of the second kind  $\Gamma^i_{ml}$ is represented as 
%%
\begin{align}
 &\Gamma^1_{ml} = 
 \begin{pmatrix}
    \dfrac{2XY^2}{\delta^2} & \dfrac{-Y(1+Y^2)}{\delta^2} & \dfrac{\delta_S}{r} \\
    \dfrac{-Y(1+Y^2)}{\delta^2} & 0 & 0  \\ 
    \dfrac{\delta_S}{r} & 0 & 0
  \end{pmatrix}, \nonumber \\
  &\Gamma^2_{ml} = 
  \begin{pmatrix}
     0 & \dfrac{-X(1+X^2)}{\delta^2} & 0 \\
     \dfrac{-X(1+X^2)}{\delta^2} & \dfrac{2X^2 Y}{\delta^2} & \dfrac{\delta_S}{r}  \\ 
     0 & \dfrac{\delta_S}{r} & 0
   \end{pmatrix},  \\
 &\Gamma^3_{ml} = \delta_S \dfrac{r(1+X^2)(1+Y^2)}{\delta^4}
   \begin{pmatrix}
     -(1+X^2) & XY & 0 \\
      XY & -(1+Y^2) & 0  \\ 
      0 & 0 & 0
    \end{pmatrix}, \nonumber
\label{eq:ChristoffelSymbol2ndKind_cubedsphere}
\end{align}
%%
where $\delta_S$ is a switch for shallow atmosphere approximation. 


The components of angular velocity vector included in the Coriolis terms
are given as
%%
\begin{align}
  \Omega^1 = 0, \;\;
  \Omega^2 = \delta_S \dfrac{\omega \delta}{r(1+Y^2)}, \;\;
  \Omega^3 = \omega \dfrac{Y}{\delta}, \;\;\;\;\;\; &\text{for the equatorial panels,} \nonumber \\
  \Omega^1 = - \delta_S \dfrac{s\omega X\delta}{r(1+X^2)}, \;\;
  \Omega^2 = - \delta_S \dfrac{s\omega Y\delta}{r(1+Y^2)}, \;\;
  \Omega^3 = \dfrac{s\omega}{\delta}, \;\;\;\;\;\;  &\text{for the polar panels}, 
%\label{eq:angular_vel_vec_cubedsphere}
\end{align}
%%
where 
$\omega$ is the angular velocity of the planet
and an index $s$ has a value of 1 and -1 for the Northern and Southern polar panels, respectively. 


\subsection{Vertical coordinates}
%%
To treat the topography, 
we adopt the traditional terrain-following coordinate \citep{PHILLIPS1957,GALCHEN1975209} as a general vertical coordinate. 
The vertical coordinate transformation can be expressed as 
%%
\begin{align}
    \zeta = z_T \dfrac{z-h}{z_T - h},   
\end{align}
%%
where
$z$ is the height coordinate, 
$z_T$ is the top height of computational domain (we assume it is a constant value), 
and $h$ is the surface height. 
The corresponding Jacobian and metric tensor can be represented as   
%%
\begin{align}
     \sqrt{G_v} = 1 - \dfrac{h}{z_T}, \;\;
    \sqrt{G_v} G_v^{13} = \left(\dfrac{\zeta}{z_T} - 1 \right) \dfrac{\partial h}{\partial \xi}, \;\;     
    \sqrt{G_v} G_v^{23} = \left(\dfrac{\zeta}{z_T} - 1 \right) \dfrac{\partial h}{\partial \eta}, 
\end{align}
%%
respectively. 



%%%%%%%%%%%%%%%%%%%%%%%%%%%%%%%%%%%%%%%%%%%%%%%%
\section{Governing equations for atmospheric dynamical core}
%%
As the governing equations, 
we adopt the three-dimensional, 
fully compressible nonhydrostatic equations based on the flux form \citep [e.g.,][]{ullrich2012dynamical}. 
The compact form of the governing equations 
can be written as
%%
\begin{align}
    \dfrac{\partial \bm{q}}{\partial t} 
    &+ \dfrac{\partial \left[\bm{f}(\bm{q})+\bm{f}_\textrm{SGS}(\bm{q},\nabla \bm{q}) \right]}{\partial \xi} 
    + \dfrac{\partial \left[\bm{g}(\bm{q})+\bm{g}_\textrm{SGS}(\bm{q},\nabla \bm{q}) \right]}{\partial \eta} 
    + \dfrac{\partial \left[\bm{h}(\bm{q})+\bm{h}_\textrm{SGS}(\bm{q},\nabla \bm{q}) \right]}{\partial \zeta} \nonumber \\
    &= \bm{S}(\bm{q}) 
    + \bm{S}_\textrm{SGS}(\bm{q},\nabla \bm{q}),
\label{eq:GovernEq_NonhydrostaticDynCore}
\end{align}
%%
Here, 
$\bm{q}$ is the solution vector defined as
%%
\begin{equation}
  \bm{q} 
  = \left( \sqrt{G} \rho', \sqrt{G} \rho u^\xi, \sqrt{G} \rho u^\eta, \sqrt{G} \rho u^\zeta, \sqrt{G} (\rho \theta)', \rho q_{*} \right)^T,  
\label{eq:solution_vector}
\end{equation}
%%
where 
$\rho, \theta$ denote the density and potential temperature defined later, respectively. 
$q_*$ represents the mass concentration of each material such as 
water vapor ($q_v$), liquid water ($q_l$), and solid water ($q_s$). 
The mass concentrations should be the relation as 
%%
\begin{align}
   q_d + \sum_{*=v,l,s} q_* = 1, 
\end{align}
%%
where $q_d$ is the mass concentration of dry air. 
To accurately treat nearly balanced flows, 
the density $\rho$ and pressure $p$ (thus $\rho \theta$) are decomposed 
as $q(\xi,\eta,\zeta,t) = q_\textrm{hyd}(\xi,\eta,\zeta) + q'(\xi,\eta,\zeta,t)$,  
where $q_\textrm{hyd}$ denotes a variable satisfying the hydrostatic balance and $q'$ denotes the deviation. 


In Eq.\,\eqref{eq:GovernEq_NonhydrostaticDynCore}, 
$\bm{f}(\bm{q})$, $\bm{g}(\bm{q})$, and $\bm{h}(\bm{q})$ are inviscid fluxes 
in the $\xi$, $\eta$, and $\zeta$ directions, respectively. 
The horizontal inviscid fluxes are represented as
%%
\begin{equation}    
\bm{f}(\bm{q})
    =\begin{pmatrix}
    \sqrt{G} \rho u^\xi \\
    \sqrt{G} (\rho u^\xi u^\xi  + G^{11}_h p') \\
    \sqrt{G} (\rho u^\eta u^\xi  + G^{21}_h p') \\
    \sqrt{G} \rho u^\zeta u^\xi  \\
    \sqrt{G} \rho \theta u^\xi \\
    \sqrt{G} \rho q_* u^\xi 
    \end{pmatrix}, \;\;\;
\bm{g}(\bm{q})
    =\begin{pmatrix}
    \sqrt{G} \rho u^\eta \\
    \sqrt{G} (\rho u^\xi u^\eta  + G^{12}_h p') \\
    \sqrt{G} (\rho u^\eta u^\eta  + G^{22}_h p') \\
    \sqrt{G} \rho u^\zeta u^\eta  \\
    \sqrt{G} \rho \theta u^\eta \\
    \sqrt{G} \rho q_* u^\eta
    \end{pmatrix},
\label{eq:horizontal_invisflux}
\end{equation}
%%
and the vertical inviscid fluxes are represented as
\begin{equation}   
\bm{h}(\bm{q})
    =\begin{pmatrix}
    \sqrt{G} \rho \widetilde{u^\zeta} \\
    \sqrt{G} [\rho u^\xi \widetilde{u^\zeta}  + (G^{13}_v G^{11}_h + G^{23}_v G^{12}_h) p'] \\
    \sqrt{G} [\rho u^\eta \widetilde{u^\zeta} + (G^{13}_v G^{21}_h + G^{23}_v G^{22}_h)  p'] \\
    \sqrt{G} \rho u^\zeta \widetilde{u^\zeta}  + \sqrt{G}_h p' \\
    \sqrt{G} \rho \theta \widetilde{u^\zeta} \\
    \sqrt{G} \rho q_* \widetilde{u^\zeta}
    \end{pmatrix}. 
\label{eq:vertical_invisflux}
\end{equation}
%%


Furthermore, 
$\bm{S}(\bm{q})$ in Eq.\,\eqref{eq:GovernEq_NonhydrostaticDynCore} represents the source terms as 
\begin{align}
\bm{S}(\bm{q})
=
  \begin{pmatrix}
    \sqrt{G} S_{\rho,{\rm phy}} \\
    \sqrt{G} ( F_H^1 + F_M^1 + F_C^1 ) + \sqrt{G} S_{\rho u^\xi,{\rm phy}} \\
    \sqrt{G} ( F_H^2 + F_M^2 + F_C^2 ) + \sqrt{G} S_{\rho u^\eta\,{\rm phy}} \\
    \sqrt{G} ( F_\textrm{buo} + F_C^3 ) + \sqrt{G} S_{\rho u^\zeta,{\rm phy}} \\
    \sqrt{G} S_{\rho \theta,{\rm phy}} \\
    \sqrt{G} S_{\rho q_*,{\rm phy}}
  \end{pmatrix},   
\end{align}
%%
where 
$F_H^i$ for $i=1,2$ are the horizontal pressure gradient terms with hydrostatic balance 
and can be expressed as 
%%
\begin{align}
  F_H^i= - \dfrac{G_h^{im'}}{\sqrt{G_v}} \left[ 
        \dfrac{\partial (\sqrt{G_v} p_\textrm{hyd})}{\partial \xi^{m'}} 
      + \dfrac{\partial (G_v^{m'3} \sqrt{G_v} p_\textrm{hyd})}{\partial \xi^3}
    \right]. 
\end{align}
%%
Here, note that $m'=1,2$. 
$F_M^i$ is the source terms due to the horizontal curvilinear coordinate as 
%%
\begin{align}
  F_M^i=-\Gamma^i_{ml} (\rho u^m u^l + G^{ml} p'), 
\end{align}
%%
where $m,l$ take values of $1,2,3$.
$F_C^i$ is the the Coriolis terms as 
%%
\begin{align}
  F_C^i=- G^{im} \epsilon_{jml} 2\Omega^m \rho u^l,
\end{align}
%%
where $\epsilon_{jkl}$ is the three rank Levi--Civita tensor 
and $\Omega^m$ are the components of angular velocity vector. 
$F_\textrm{buo}$ is the buoyancy term as 
%%
\begin{align}
 F_\textrm{buo}=- \rho' \left(\dfrac{a}{r}\right)^2 g,   
\end{align}
%%
where $r$ is the radial coordinate, 
$a$ is the planetary radius, 
and $g$ is the standard gravitational acceleration. 
In Eq.~(\ref{eq:GovernEq_NonhydrostaticDynCore}), 
the terms with subscript SGS are associated with a turbulent model;  
$\bm{f}_\textrm{SGS}(\bm{q},\nabla\bm{q})$, $\bm{g}_\textrm{SGS}(\bm{q},\nabla\bm{q})$, 
and $\bm{h}_\textrm{SGS}(\bm{q},\nabla\bm{q})$ are the parameterized eddy fluxes 
while  
$\bm{S}_\textrm{SGS}(\bm{q},\nabla\bm{q})$ are the source terms with the curvilinear coordinates. 
The terms associated with the turbulent model are detailed in Sect.\,\ref{sec:turbulence}. 
On the other hand, 
$\bm{S}_{*,\textrm{phys}}$ represents the terms with physical processes 
such as cloud, radiation, and surface processes. 


To consider the moist thermodynamics, 
we introduce a potential temperature defined as 
%%
\begin{align}
  \theta = T \left(\dfrac{P_0}{p}\right)^{R^*/C_p^*}, 
\end{align}
where 
$T$ is the temperature, 
$P_0$ is a constant pressure. 
The gas constant $R^*$ and 
the specific heat at constant volume $C_p^*$ are defined as 
\begin{align}
  C_p^* = \sum_{*=d,v,l,s} q_{*} C_{p,*}, \;\;\;
  R^* = \sum_{*=d,v,l,s} q_{*} R_{*}. 
\end{align}
To close the equation systems in Eq.\,\eqref{eq:GovernEq_NonhydrostaticDynCore}, 
the pressure $p$ is calculated using
%%
\begin{align}
    p=P_0 \left(\dfrac{R^*}{P_0} \rho \theta  \right)^{\frac{C_{p}^*}{C_{v}^*}}, 
\end{align}
%%
where $C_v^* = C_p^* - R^*$.  
%The actual values for these constants are provided in Table \ref{tb:symbols}. 
In terms of $C_p^*$ and $R^*$, 
a diabatic heating contribution $H_{\rho \theta}$ included in $\bm{S}_{\rho \theta,\textrm{phys}}$ can be written as
%%
\begin{align}
  H_{\rho \theta} = \dfrac{1}{C_p^*} \left(\dfrac{P_0}{p}\right)^{\frac{R^*}{C_p^*}} Q,
\end{align}
%%
where $Q$ is the diabatic heating with the unit J/(m$^3$s).


When the traditional approximation is applied to the governing equations in global model, 
$\delta_S$ should be set to zero.
