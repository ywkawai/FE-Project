\section{Turbulence}\label{sec:turbulence}
{\bf \Large
\begin{tabular}{ccc}
\hline
  Corresponding author & : & Yuta Kawai\\
\hline
\end{tabular}
}

\subsection{Smagorinsky-type model}
%% 
As a turbulent model, 
this subsection describes 
a Smagorinsky--Lilly type model \citep{smagorinsky1963general,lilly1962numerical} 
that considered the stratification effect \citep{brown1994largeeddy}. 
As a spatial filter, 
the Favre-filtering \citep{favre1983turbulence} was used.  
We do not explicitly denote the symbol representing the spatial filter 
because the filtering approach is essentially the same as that explained in Appendix A of \cite{KT2023NumAccuracyDG}. 
The difficulties in the derivation of viscous and diffusion terms 
are caused by the gradient of vector quantities 
and the spatial divergence with the non-orthogonal basis
because the manipulations grow increasingly complex. 
However, 
previous studies that utilized tensor analysis help us provide a systematic derivation \citep[e.g.,][]{ullrich_global_2014,Rancic2017Nonhydro}. 
In the absence of a vertical coordinate transformation, 
the parameterized fluxes with the turbulent model can be represented in the general curvilinear coordinates as 
%%
\begin{align}    
&\bm{f}_\textrm{SGS}(\bm{q},\nabla \bm{q})
    =\begin{pmatrix}
    0 \\
    -\sqrt{G} \rho \tau^{11}  \\
    -\sqrt{G} \rho \tau^{12}  \\
    -\sqrt{G} \rho \tau^{13}  \\
    -\sqrt{G} \rho \tau_*^1
  \end{pmatrix}, \;\;\;
\bm{g}_\textrm{SGS}(\bm{q},\nabla \bm{q})
    =\begin{pmatrix}
    0 \\
    -\sqrt{G} \rho \tau^{21}  \\
    -\sqrt{G} \rho \tau^{22}  \\
    -\sqrt{G} \rho \tau^{23}  \\
    -\sqrt{G} \rho \tau_*^2
    \end{pmatrix}, \\
&\bm{h}_\textrm{SGS}(\bm{q},\nabla \bm{q})
    =\begin{pmatrix}
    0 \\
    -\sqrt{G} \rho \tau^{31}  \\
    -\sqrt{G} \rho \tau^{32}  \\
    -\sqrt{G} \rho \tau^{33}  \\
    -\sqrt{G} \rho \tau_*^3
    \end{pmatrix}, 
\end{align}
%%%%
and 
the source term can be given by 
%%%
\begin{equation}
  \bm{S}_\textrm{SGS}(\bm{q},\nabla \bm{q}) = \begin{pmatrix}
    0 \\
    -\sqrt{G} \Gamma^1_{ml}\rho \tau^{ml}  \\
    -\sqrt{G} \Gamma^2_{ml}\rho \tau^{ml}  \\
    -\sqrt{G} \Gamma^3_{ml}\rho \tau^{ml}  \\
    0
    \end{pmatrix}.
\end{equation}
%%
In the equations, 
$\tau^{ij}$ is the contravariant components of parameterized eddy viscous flux tensor 
($i=1,2,3$ and $j=1,2,3$) and can be written as 
%%
\begin{eqnarray}
  \tau^{ij} = - 2 \nu_\textrm{SGS}
   \left( 
    S^{ij} - \dfrac{G^{ij}}{3} D
   \right)
   - \dfrac{2}{3} G^{ij} K_\textrm{SGS}, 
\end{eqnarray} 
%%
where
$S^{ij}$ is the strain velocity tensor, 
$\nu_\textrm{SGS}$ is the eddy viscosity, 
$D$ is the divergence of the three-dimensional velocity, 
and 
$K_\textrm{SGS}$ is the SGS kinetic energy. 
The strain velocity tensor is represented as  
%%
%% grad u = [du/dx^i] a^i
%         = [u^k,i a_k a^i]
%         = [u^k,i a_k G^im a_m]
%         = [G^im u^k,i a_k a_m ] 
%         = [G^im u^j,m] a_i a_j
\begin{equation}
  S^{ij} = \dfrac{1}{2} 
  \left( G^{im} \dfrac{\partial u^j_{,m}}{\partial \xi^m} + G^{jm} \dfrac{\partial u^i_{,m}}{\partial \xi^m} \right), 
\end{equation}
%%
using the covariant derivative of the contravariant velocity component 
%%
\begin{equation}
  u^i_{,j} = \dfrac{\partial u^i}{\partial \xi^j} + u^m \Gamma^i_{jm}.
\end{equation}
%%
The eddy viscosity is written as 
%%
\begin{equation}
  \nu_\textrm{SGS} 
 = C_s \Delta_\textrm{SGS} |S|, 
\end{equation}
%%
where 
$C_s$, $\Delta_\textrm{SGS}$, 
and $|S|$ represent the Smagorinsky constant, 
the filter length, 
and the norm of strain tensor defined as $\sqrt{2 G_{im}G_{jn}S^{ij}S^{mn}}$, respectively.
The parameterized eddy diffusive flux can be written 
as 
%%
\begin{eqnarray}
  \tau_*^{i} = - \nu^*_\textrm{SGS} G^{ij} \dfrac{\partial \theta}{\partial \xi^j}, 
\end{eqnarray}
%%
where 
$\nu^*_\textrm{SGS}$ is the eddy diffusion coefficient.  
For further details of the turbulent model, 
refer to Sect.\,2.2 of \cite{nishizawa2015influence}. 

