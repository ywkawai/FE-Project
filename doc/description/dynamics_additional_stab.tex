\section{Additional stabilization}
%%
\subsection{Modal filtering}
%%

For high-order DGM, 
numerical instability is likely to occur in advection-dominated flows 
because the numerical dissipations with the upwind numerical fluxes weaken. 
Furthermore, 
we adopted a collocation approach due to its computational efficiency. 
One drawback is that 
the aliasing errors with evaluations of the nonlinear terms can drive numerical instability. 
To suppress this numerical instability, 
a modal filter was used as an additional stabilization mechanism. 
The filter matrix for the three-dimensional problem can be obtained as 
%%
\begin{equation}
  \mathscr{F} =  V^{\rm 3D} C^{\rm 3D} V^{\rm 3D},   
\end{equation}
%%
where 
$V^{\rm 3D}$ represents the Vandermode matrix associated with the LGL interpolation nodes (in Eq.\,\eqref{eq:polynomial_expression}) 
and  
$C^{\rm 3D}$ represents the diagonal cutoff matrix. 
The entries of $C^{\rm 3D}$ are defined as
%%
\begin{equation}
  C^{\rm 3D}_{(m_1,m_2,m_3),(m'_1,m'_2,m'_3)} 
 = 
 \delta_{m_1,m'_1} \sigma^h_{m_1} \;
 \delta_{m_2,m'_2} \sigma^h_{m_2} \;
 \delta_{m_3,m'_3} \sigma^v_{m_3}, 
\end{equation}
%%
where
$\sigma^h_i$ and $\sigma^v_i$ represent 
the decay coefficient for the one-dimensional horizontal and vertical modes $i$, respectively. 
Based on \cite{hesthaven2007nodal}, 
a typical choice of the coefficient for mode $i$ is provided 
with an exponential function as 
%%
\begin{align}
 \sigma_i = \begin{cases}
   1 \;\; &{\rm if}\;\;  0 \leq i \leq p_c  \\
   \exp \left[-\alpha_m \left( \dfrac{i-p_c}{p-p_c} \right)^{p_m} \right] \;\; &{\rm if} \;\; p_c \leq i \leq p, 
 \end{cases}
\label{eq:modal_filter_coefficient}
\end{align}
%%
where 
$p_c$, $p_m$, and $\alpha_m$ represent the cutoff parameter, 
the order of the filter, and the non-dimensional decay strength, respectively. 
In this study, $p_c$ was considered 0. 
We applied the filter $\mathscr{F}$ to the solution vector $\bm{q}$ (in Eq.\,\eqref{eq:solution_vector})
at the final stage of the RK scheme with a timestep $\Delta t$. 
Then, the decay time scale for the highest mode can be regarded as approximately equal to $\Delta t/\alpha_m$. 
We set the order $p_m$, and decay coefficient $\alpha_m$ such that 
the strength of filter should ensure numerical stability while being as weak as possible. 


\subsection{Buoyancy term}
%%
The balance between the pressure gradient and buoyancy terms should be carefully treated  
in the discrete momentum equation \citep[e.g.,][]{Blaise2016StablizationDG,ORGIS2017461}. 
In the above formulation, 
because a different discretization space is used between the terms, 
a numerical imbalance is possible and may cause spurious oscillations, which can destabilize the simulations. 
To avoid this incompatibility, 
the vertical polynomial order for the density in the buoyancy term was reduced by one 
following \cite{Blaise2016StablizationDG}. 
