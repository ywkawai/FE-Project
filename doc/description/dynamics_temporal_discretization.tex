\section{Temporal discretization}
%%
The semi-discretized equations in Eq.\,\eqref{eq:semidiscrete_eqs} can be represented  
as the following ordinary differential equation (ODE) system  
%%
\begin{equation}
  \dfrac{d \bm{q}}{d t} = \mathcal{S}(\bm{q},\nabla\bm{q}) + \mathcal{F}(\bm{q},\nabla\bm{q}), 
\end{equation}
%%
where 
$\mathcal{S}(\bm{q},\nabla\bm{q})$ and $\mathcal{F}(\bm{q},\nabla\bm{q})$ 
represent the tendencies with slow and fast contributions, respectively.  
This study adopted Runge--Kutta (RK) schemes to solve the ODE system 
from $t=n\Delta t$ to $t=(n+1)\Delta t$, 
where $\Delta t$ is the time step and $n$ is a natural number. 
In this subsection, 
we describe two approaches for temporal discretization, 
namely, 
horizontal explicit and vertical implicit (HEVI) 
and horizontal explicit and vertical explicit (HEVE) approaches. 


% We introduce two types of Courant number, 
% which are used to explain timestep settings.  
% For the horizontal advection test, 
% the advective Courant number associated with the horizontal wind is defined as $C_{r,\textrm{adv}}=U_0\Delta t/\Delta_{h}$, 
% where $U_0$ is the representative wind speed. 
% For other numerical experiments,  
% the acoustic Courant number associated with the sound wave propagation is defined as $C_{r,c_s}=c_s\Delta t/\Delta$, 
% where $\Delta$ is the grid spacing; 
% In particular, for the HEVI approach, $\Delta=\Delta_{h}$.  

\subsubsection{HEVI approach}
\label{sssec:temporal_discretization_hevi}
%%
%%
If the aspect ratio of horizontal grid spacing to its vertical counterpart is large, 
it is impractical to use fully explicit temporal schemes 
because 
the vertically propagating sound waves severely restrict the timestep. 
A strategy to avoid computational cost in such case is the HEVI approach. 
The terms corresponding to vertical dynamics with a fast time-scale 
are evaluated using an implicit temporal scheme, 
while 
the remaining terms are evaluated using an explicit temporal scheme. 
This procedure is regarded as a framework of 
implicit-explicit (IMEX) time integration scheme
\citep{Bao2015HEVI,Gardner2018IMEX}. 
%  Durran and Blossey (2012); Ullrich and Jablonowski (2012); Giraldo et al. (2013); Lock et al. (2014); Weller et al. (2013). 
% The remainder of this discussion considers the analyses described in Lock et al. (2014) and Weller et al. (2013).
General formulation of IMEX RK scheme \citep[e.g.,][]{ASCHER1997IMEX} with $\nu$ stages can be represented as 
%%
\begin{eqnarray}
 \bm{q}^{(s)} &=& \bm{q}^n 
                 + \Delta t \sum_{s'=1}^{s-1} a_{ss'} \mathcal{S}(t+c_{s'}\Delta t, \bm{q}^{(s')})
                 + \Delta t \sum_{s'=1}^{s} \tilde{a}_{ss'} \mathcal{F}(t+\tilde{c}_{s'}\Delta t, \bm{q}^{(s')}) 
 \;\;\;\; \textrm{for}\;\;\; s=1, \dots, \nu  \nonumber \\
 \nonumber \\
 \bm{q}^{n+1} &=& \bm{q}^n 
                + \Delta t \sum_{s=1}^{\nu} b_{s} \mathcal{S}(t+c_{s}\Delta t, \bm{q}^{(s)})
                + \Delta t \sum_{s=1}^{\nu} \tilde{b}_{s} \mathcal{F}(t+\tilde{c}_{s}\Delta t, \bm{q}^{(s)}), 
\end{eqnarray}
%%
where 
$a_{ss'}$, $b_s$, and $c_{s}$ define the explicit temporal integrator, 
while $\tilde{a}_{ss'}$, $\tilde{b}_s$, and $\tilde{c}_{s'}$ define the implicit temporal integrator; 
$c_s=\sum_{s'=1}^{s-1} a_{ss'}$ and $\tilde{c}_s = \sum_{s'=1}^{s-1} \tilde{a}_{ss'}$ 
represents time when slow and fast terms are evaluated, respectively. 
These coefficients are compactly represented using ``double Butcher tableaux'', 
as shown in Table \ref{tb:Butcher_tbl_IMEXRK}.  
Note that, in the table of the explicit part, 
$\mathscr{A}=\{a_{ss'}\}$ with $a_{ss'}=0$ for $s' \ge s$. 
On the other hand, 
for the implicit part,  
$\tilde{\mathscr{A}}=\{\tilde{a}_{ss'}\}$ with $\tilde{a}_{ss'}=0$ for $s' > s$ 
in the case of the diagonally implicit RK scheme. 


The terms associated with 
vertical mass flux, 
vertical pressure gradient, 
vertical flux of potential temperature, 
and buoyancy in Eq.\,\eqref{eq:GovernEq_NonhydrostaticDynCore} were treated as fast terms, 
whereas the other terms were treated as slow terms. 


In SCALE-DG, 
several implicit and explicit (IMEX) RK schemes in Table \ref{tb:HEVI_temporal_integ_choice} can be available. 
To minimize contaminating the spatial accuracy of high-order DGM by temporal errors present in low-order HEVI scheme, 
a third-order scheme proposed by \cite{kennedy2003additive} was adopted in our previous studies \citep{KT2025SCALEDG};  
it includes four explicit and three implicit evaluations. 
The corresponding double Butcher tableaux are given in Table \ref{tb:Butcher_tbl_IMEXRK}. 

%t
\begin{table}[t]
\caption{Implicit and Explicit (IMEX) Runge--Kutta schemes available in HEVI temporal integration. 
For the number of RK stages, (I,E) represents implicit and explicit parts, respectively.}
%% Explicit part
\begin{tabular}{l|cccc}
%\tophline
\hline
Abbrev. & Order & Num. of stages (I,E) & Note & Reference \\
\hline
IMEX\_ARK232 & 2 & (2,3) &  & \cite{Giraldo2013} \\
\hline
IMEX\_ARK324 & 3 & (3,4) &  & \cite{kennedy2003additive} \\
\hline
\end{tabular}
%%%%%
%\belowtable{} % Table Footnotes
\label{tb:HEVI_temporal_integ_choice}
\end{table}


In the implicit part of each stage, 
the corresponding nonlinear equation system is solved using Newton's method. 
In each iteration, the linearized equation system is solved. 
Obtaining accurate solutions of the nonlinear equation system generally requires numerous iterations. 
However, 
this study performed a single iteration in Newton's method (i.e., Rosenbrock approach), 
significantly reducing the computational cost. 
Similar approach has been used in previous studies \citep{Ullrich2012Resenbrock}. 
In the case of the collocation approach, 
because the horizontal dependency between all nodes within an element vanishes,  
the vertical implicit evaluation can be parallelly performed at each horizontal node. 


For the case of HEVI, 
the volume and surface integrations in Eqs.~(\ref{eq:differential_and_lifting_matrix}) and (\ref{eq:mass_matrix}) 
were evaluated using inexact integration with the LGL nodes. 
Consequently, $\bm{M}$ and $\bm{L}_{\partial \Omega_{e,3}}$ became diagonal matrices, 
which further simplified the matrix structure associated with the vertical spatial operator. 


\subsubsection{HEVE approach}
\label{sssec:temporal_discretization_heve}
%%
%%
When we consider a horizontal grid spacing with O(10 m) such as in LES, 
the ratio of horizontal to vertical grid spacing approaches unity. 
The advantages of HEVI approach decrease. 
Thus, 
it is suitable to adopt a fully explicit temporal approach, 
referred to as HEVE approach. 
In such cases, 
RK schemes with a strong stability preserving (SSP) property \citep{gottlieb2001} are often used in combination with DGM. 
In SCALE-DG, several RK schemes in Table \ref{tb:HEVE_temporal_integ_choice} can be available. 
The corresponding Butcher table and coefficients with Shu-Osher form are shown in Appendix \ref{sec:appendix_RK}. 
In our previous studies \citep{KT2023NumAccuracyDG,KT2025SCALEDG}, 
we adopted a ten-stage RK scheme with the fourth-order accuracy proposed by \cite{Ketcheson2008RK4o10s}. 

%t
\begin{table}[t]
\caption{Explicit Runge--Kutta schemes available in HEVE temporal integration}
%% Explicit part
\begin{tabular}{l|cccc}
%\tophline
\hline
Abbrev. & Order & Num. of stages & Note & Reference \\
\hline
ERK\_Euler & 1 & 1 & for debug & \\
\hline
ERK\_SSP\_2s2o & 2 & 2 & SSP & \cite{SHU1988439} \\
\hline
ERK\_SSP\_3s3o & 3 & 3 & SSP & \cite{SHU1988439} \\
ERK\_SSP\_4s3o & 3 & 4 & SSP & \\
ERK\_SSP\_5s3o\_2N2* & 3 & 5 & SSP & \cite{higueras2019new} \\
\hline
ERK\_RK4 & 4 & 4 & classical RK4 & \\
ERK\_SSP\_10s4o\_2N & 4 & 10 & SSP & \cite{Ketcheson2008RK4o10s} \\
\hline
\end{tabular}
%%%%%
%\belowtable{} % Table Footnotes
\label{tb:HEVE_temporal_integ_choice}
\end{table}


When using the HEVE approach, 
entries of the matrices in Eqs.\,\eqref{eq:differential_and_lifting_matrix} and \eqref{eq:mass_matrix}
were directly calculated following Sect.\,3.2 in \cite{hesthaven2007nodal}. 
