{\bf \Large
\begin{tabular}{ccc}
\hline
  Corresponding author & : & Yuta Kawai\\
\hline
\end{tabular}
}

\section{Spatial discretization}
%%
We perform the spatial discretization for Eq.\,\eqref{eq:GovernEq_NonhydrostaticDynCore}
based on a nodal DGM \citep[e.g.,][]{hesthaven2007nodal}. 
The three-dimensional computational domain $\Omega$ is divided 
using non-overlapping hexahedral elements. 
To relate the coordinates $(\xi^1,\xi^2, \xi^3)$ with the local coordinates 
$\tilde{\bm{x}} \equiv (\tilde{x}^1,\tilde{x}^2,\tilde{x}^3)$ 
in a reference element $\Omega_e$, 
we adopted a linear mapping defined as 
%%
\begin{equation}
 \tilde{x}^i = 2\dfrac{\xi^i-\xi^i_e}{h^i_{e}}, 
\label{eq:linear_mapping}
\end{equation}
%%
where 
$\xi^i_{e}$ and $h^i_e$ represent the center position and width of the element, respectively, in the $\xi^i$-direction. 


Using the tensor-product of one-dimensional Lagrange polynomials 
%%
\begin{align}
 l_{\bm{m}} (\tilde{\bm{x}}) = l_{m_1}(\tilde{x}^1) l_{m_2}(\tilde{x}^2) l_{m_3} (\tilde{x}^3),   
\end{align}
%%
a local approximated solution within each element $\Omega_e$ can be represented as
%%
\begin{equation}
  \bm{q}^e|_{\Omega_e} (\tilde{\bm{x}},t) 
  = \sum_{m_1=1}^{p+1} \sum_{m_2=1}^{p+1} \sum_{m_3=1}^{p+1} 
      \bm{Q}^e_{m_1,m_2,m_3} (t) \;\; l_{m_1}(\tilde{x}^1) l_{m_2}(\tilde{x}^2) l_{m_3} (\tilde{x}^3), 
\label{eq:polynomial_expression}
\end{equation}
%%
In Eq.\,\eqref{eq:polynomial_expression}, 
the coefficients $Q^e_{m_1,m_2,m_3}$ are the unknown degrees of freedom (DOF) 
and $p$ is the polynomial order.
In this study, the  Legendre--Gauss--Lobatto (LGL) points were used for interpolation and integration nodes. 


\subsubsection{Semi-discretized equations}
%%
By applying the Galerkin approximation to Eq.\,\eqref{eq:GovernEq_NonhydrostaticDynCore}, 
a strong form of the semi-discretized equations can be obtained as 
%%
\begin{align}
  \dfrac{D}{Dt} \int_{\Omega_e} \bm{q}^e (\tilde{\bm{x}},t) \; l_{\bm{m}} (\tilde{\bm{x}}) \; J^E \; d\tilde{\bm{x}}
  = 
   &- \sum_{j=1}^3  
    \int_{\Omega_e} \dfrac{\partial \bm{F}_j (\bm{q}^e,\bm{G})}{\partial \xi^j}  l_{\bm{m}} (\tilde{\bm{x}}) \; J^E \; d\tilde{\bm{x}} \nonumber \\
   &- 
    \int_{\partial \Omega_{e}} \left[ \hat{\bm{F}} (\bm{q}^e,\bm{G}) - \bm{F}  (\bm{q}^e,\bm{G}) \right] \cdot \bm{n} \; l_{\bm{m}} (\tilde{\bm{x}}) \; J^{\partial E} \; dS  \nonumber \\
   &+ \int_{\Omega_{e}} \left[ \bm{S} (\bm{q}^e) + \bm{S}_{\rm SGS} (\bm{q}^e,\bm{G}) \right] l_{\bm{m}} (\tilde{\bm{x}}) \; J^E \; d\tilde{\bm{x}}, 
\label{eq:semidiscrete_eqs}
\end{align}
%%
where $(\bm{F}_1,\bm{F}_2,\bm{F}_3)=(\bm{f}+\bm{f}_{\rm SGS},\bm{g}+\bm{g}_{\rm SGS},\bm{h}+\bm{h}_{\rm SGS})$ is the flux vector tensor, 
$\hat{\bm{F}}$ is the numerical flux at the element boundary $\partial\Omega_E$, 
and 
$\bm{n}$ is the outward unit vector normal to $\partial\Omega_E$; 
In the volume and surface integrals, 
$J^E$ and $J^{\partial E}$ represent the transformation Jacobian with the general curvilinear coordinates 
and local coordinates within each element. 
Note that, because of the linear mapping in Eq.\,\eqref{eq:linear_mapping}, 
the associated geometric factors such as $J^E$ and $J^{\partial E}$ have constant values  
when the volume and surface integrals are calculated. 
For the turbulent model,  
we need to evaluate 
the eddy viscous flux tensor and diffusion flux, 
which include a few gradient terms with quantities such as $\bm{\chi}=(u^\xi,u^\eta,u^\zeta,\theta,q_{v,l,s})$, 
denoted by 
%%
$\bm{G}=(\partial \bm{\chi}/\partial\xi^1,\partial \bm{\chi}/\partial\xi^2,\partial \bm{\chi}/\partial\xi^3)$ 
in Eq.\,\eqref{eq:semidiscrete_eqs}. 
The gradient discretization in the $\xi^j$-direction is given by 
%%
\begin{align}
  \int_{\Omega_e} \rho \; \bm{G}_j l_{\bm{m}} (\tilde{\bm{x}}) \; J^E \; d\tilde{\bm{x}}
  =  
  &  \int_{\Omega_e} \left[\dfrac{\partial \rho^e \bm{\chi}^e}{\partial \xi^j} - \bm{\chi}^e  \left(\dfrac{\partial \rho}{\partial \xi^j}\right)^e \right] l_{\bm{m}} (\tilde{\bm{x}}) \; J^E \; d\tilde{\bm{x}} \nonumber \\
  &+ \int_{\partial \Omega_{e}} \left( \widehat{\rho\bm{\chi}} - \rho^e \bm{\chi}^e \right)\bm{n}_{\tilde{x}^j}  \cdot \bm{n} \; l_{\bm{m}} (\tilde{\bm{x}}) \; J^{\partial E} \; dS, \label{eq:semidiscrete_grad}
\end{align}
%%
where 
$\bm{n}_{\tilde{\bm{x}}^j}$ is the unit vector in the $\tilde{x}^j$-direction 
and  
the density gradient is calculated by
%%
\begin{align}
  \int_{\Omega_e}  \left(\dfrac{\partial \rho}{\partial \xi^j}\right)^e l_{\bm{m}} (\tilde{\bm{x}}) \; J^E \; d\tilde{\bm{x}}
=  
    \int_{\Omega_e} \dfrac{\partial \rho^e}{\partial \xi^j} l_{\bm{m}} (\tilde{\bm{x}}) \; J^E \;  d\tilde{\bm{x}} 
  + \int_{\partial \Omega_{e}} (\hat{\rho} - \rho^e)\bm{n}_{\tilde{x}^j}  \cdot \bm{n} \; l_{\bm{m}} (\tilde{\bm{x}}) \; J^{\partial E} \; dS. 
\end{align}
%%


\subsubsection{Numerical flux}
%%
For the numerical flux of the inviscid terms, 
the Rusanov flux \citep{Rusanov1961} is used 
as a simple choice of the approximated Riemann solvers. 
Its numerical dissipation is provided based on the maximum absolute eigenvalue of the Jacobian matrix 
at the left and right sides of the element boundary. 
The Rusanov flux is written as 
%%
\begin{equation}
 \hat{\bm{F}}_{\rm invis} 
 = \dfrac{1}{2}\left\{ \left[ \bm{F}_{\rm invis}(\bm{q}^+) + \bm{F}_{\rm invis}(\bm{q}^-) \right]\cdot\bm{n} - \lambda_{\rm max} \left[\bm{q}^+ - \bm{q}^- \right] \right\}, 
\end{equation}
%%
where 
$\lambda_{\rm max}$ is the maximum of the absolute value of eigenvalues of the flux Jacobian in the direction $\bm{n}$, and 
$\bm{q}^-$ and $\bm{q}^+$ represent the interior and exterior values at $\partial \Omega_e$. 
Previous studies \citep[e.g.,][]{li2020development} formulated 
the Rusanov flux taken into account the horizontal and vertical coordinate transformations. 
Based on their works, 
at the element boundaries in the horizontal directions ($\xi$ and $\eta$), 
$\lambda_{\rm max}$ can be represented as  
%%
\begin{equation}
 \lambda_{{\rm max},\xi}=\left|u^\xi \right| + \sqrt{G_h^{11}} c_s, 
\;\;\; 
\lambda_{{\rm max},\eta}=\left|u^\eta \right| + \sqrt{G_h^{22}} c_s, 
\end{equation}
%%
where $c_s=[(C_p/C_v) RT]^{1/2}$ is the speed of sound wave.
For the vertical direction $\zeta$, 
$\lambda_{\rm max}$ can be represented as 
%%%
\begin{equation}
  \lambda_{{\rm max},\zeta} 
    = \left|\widetilde{u^\zeta} \right| 
    + \left[ 1/\sqrt{G_v} + G_v^{13}G_X + G_v^{23}G_Y \right]^{1/2} c_s, 
\end{equation}
%%%
where
$G_X=G_v^{13}G_h^{11} + G_v^{23}G_h^{12}$ 
and 
$G_Y=G_v^{13}G_h^{21} + G_v^{23}G_h^{22}$. 


We adopt the central flux  
as the numerical flux of the gradient $\bm{G}$ and 
the SGS fluxes $(\bm{f}_{\rm SGS},\bm{g}_{\rm SGS},\bm{h}_{\rm SGS})$ 
with the turbulent model. 


\subsubsection{Matrix form of semi-discretized equation}
%%
When the same nodes are used for interpolation and integration (i.e., collocation), 
a matrix form of Eqs.\,\eqref{eq:semidiscrete_eqs} and \eqref{eq:semidiscrete_grad} can be obtained as   
%%
\begin{align}
    \dfrac{D \bm{q}^e}{Dt} 
    =&
      - \sum_{j=1}^{3} d_j D_{\tilde{x}^j} \bm{F}_j  (\bm{q}^e,\bm{G})
      - \sum_{f=1}^{6} s_{\partial \Omega_{e,f}} L_{\partial \Omega_{e,f}} \left[ \hat{\bm{F}} (\bm{q}^e,\bm{G}) 
                 - \bm{F}  (\bm{q}^e,\bm{G}) \right] 
               \cdot \bm{n} \nonumber \\
      &+ \bm{S} (\bm{q}^e) + \bm{S}_{\rm SGS} (\bm{q}^e,\bm{G}), 
\label{eq:semidiscrete_eqs_matform} 
\\
%%
  \rho \; \bm{G}_j
  =&
  d_j D_{\tilde{x}^j} (\rho^e \bm{\chi}^e) 
    - \bm{\chi}^e  \left(\dfrac{\partial \rho}{\partial \xi^j}\right)^e
  + \sum_{f'=1}^2 s_{\partial \Omega_{e,f'}} L_{\partial \Omega_{e,f'}} \left( \widehat{\rho\bm{\chi}} - \rho^e \bm{\chi}^e \right)\bm{n}_{\tilde{x}^j}  \cdot \bm{n}, 
\label{eq:semidiscrete_eqs_aux_matform}
\end{align}
%%
where 
$D_{\tilde{x}^j}$ represents the differential matrix for the $\tilde{x}^j$-direction; 
$L_{\partial \Omega_{e,f}}$ represents the lifting matrix with the surface integral for the $f$-th element surface, 
and $L_{\partial \Omega_{e,f'}}$ represents the same for the $f'$-th element surface in the gradient operator for the $\tilde{x}^j$-direction. 
The components of these matrices are given as 
%%
\begin{equation}
 \left( D_{\tilde{x}^j} \right)_{\bm{m},\bm{m}'} 
 = M^{-1} \int_{\Omega_e}  l_{\bm{m}} \dfrac{\partial l_{\bm{m}'}}{\partial \tilde{x}_j}  \; d\tilde{\bm{x}}, 
 \;\;\;\;
 \left( L_{\partial \Omega_e,j} \right)_{\bm{m},{\bm{m}'}} 
 = M^{-1} \int_{\partial \Omega_{e,j}}  l_{\bm{m}} l^{\partial \Omega_{e,j}}_{\bm{m}'} dS, 
\label{eq:differential_and_lifting_matrix}
\end{equation}
%%
where $M$ denotes the mass matrix and is given by
%%
\begin{equation}
 M_{\bm{m},\bm{m}'} 
  = \int_{\Omega_e}  l_{\bm{m}} l_{\bm{m}'}  \; d\tilde{\bm{x}}.
\label{eq:mass_matrix}
\end{equation}
%%
The density gradient term is calculated as  
%%
\begin{equation}
\left(\dfrac{\partial \rho}{\partial \xi^j}\right)^e
= 
d_j  D_{\tilde{x}^j} \rho^e
- \sum_{f'=1}^2 s_{\partial \Omega_{e,f'}} L_{\partial \Omega_{e,f'}} \left( \widehat{\rho} - \rho^e \right)\bm{n}_{\tilde{x}^j}  \cdot \bm{n}.   
\label{eq:semidiscrete_density_gradient_matform}
\end{equation}
%%
Note that, 
in Eqs.\,\eqref{eq:semidiscrete_eqs_matform}, \eqref{eq:semidiscrete_eqs_aux_matform}, and \eqref{eq:semidiscrete_density_gradient_matform}, 
$d_j=\partial \tilde{x}^j/\partial \xi^j$ 
and 
$s_{\partial \Omega_e,f'} =J_{\partial \Omega_{e,f'}}/J^E$ 
are constant values in the volume and surface integrals, respectively. 
% We changed the calculation method of mass and lifting matrices depending on temporal discretization; 
% This is detailed in Sect.\,\ref{ssec:temporal_discretization}. 
