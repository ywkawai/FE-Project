This chapter explains how to compile \scaledg
and the minimum computational requirements for their execution.

\section{System environment} \label{sec:req_env}
%====================================================================================
% \subsubsection{Recommended Hardwares}

%   Although the necessary hardware depends on an experiment to be performed,
% the following are the specifications for conducting the tutorial as in Chapters \ref{chap:tutorial_ideal} and \ref{chap:tutorial_real}.

%   \begin{itemize}
%     \item {\bf CPU} :
%     The system need to have more than or equal to two physical cores for the ideal experiment in the tutorial, and more than or equal to four for the real atmospheric experiment.
%     \item {\bf Memory} :
%     The system needs 512 MB and 2 GB memory
%     for the ideal experiment and the real atmospheric experiment, respectively.
%     Note that this applies to the case involving the use of double-precision floating points.
%     \item {\bf HDD} : The system needs to have 3 GB of free disk space in the real atmospheric experiment.
%   \end{itemize}


\subsubsection{Required Softwares}

  \begin{itemize}
  \item {\bf OS} : Linux OS, Mac OS
  %Refer to Table \ref{tab:compatible_os} for other OSs currently supported.
  \item {\bf Compiler} : C, Fortran
  \end{itemize}

Since Fortran source codes of \scalelib library and FE-Project are written in FORTRAN 2008 standard syntax, the compiler must support it.
% For example, GNU gfortran version X.X or later versions is necessary.
Refer to Table \ref{tab:compatible_compiler} for compilers already confirmed as supported.


\begin{table}[tb]
\begin{center}
\caption{Compiler already checked}
\begin{tabularx}{150mm}{|l|X|} \hline
 \rowcolor[gray]{0.9} Name of compiler &  \\ \hline
  GNU (gcc/gfortran)    & Version 11.4.0 or later is supported. \\ \hline
  Intel HPC Toolkit (icc/ifx)     & Version 2025.0 or later is supported. \\ \hline
  Fujitsu on Fugaku (fccpx/frtpx) & tcsds-1.2.41 or later is supported. \\ \hline
%  NVIDIA HPC SDK (nvcc/nvfortran) &  \\ \hline
\end{tabularx}
\label{tab:compatible_compiler}
\end{center}
\end{table}



\subsubsection{Required libraries}\label{sec:inst_env}

The required external libraries are described below:
\begin{itemize}
%\item HDF5 Library (\url{https://www.hdfgroup.org/HDF5/})
\item {\netcdf} Library (\url{http://www.unidata.ucar.edu/software/netcdf/})
\item MPI Library (e.g., openMPI \url{http://www.open-mpi.org/})
\item LAPACK ( \url{http://www.netlib.org/lapack/} )
\item SCALE library ( \url{https://scale.riken.jp} )
\end{itemize}

To install several libraries, 
you can use distributed binary packages for Linux OS and Mac OS.

For the MPI library, the MPI 1.0/2.0 protocol should be supported.  
Refer to Table \ref{tab:compatible_mpi} for MPI libraries already confirmed as supported.

\begin{table}[tb]
\begin{center}
\caption{MPI libraries already checked}
\begin{tabularx}{150mm}{|l|X|} \hline
 \rowcolor[gray]{0.9} Name of MPI library &  \\ \hline
  Open MPI               & Version 4.1.2 or later is supported. \\ \hline
  Intel MPI             & Version 2021.14 or later is supported.\\ \hline
\end{tabularx}
\label{tab:compatible_mpi}
\end{center}
\end{table}


\subsubsection{Visualization tools}

In this subsection, drawing tools that can draw the initial conditions, boundary data, and the simulation results with \scaledg are introduced.

The GPhys and \grads are used for a quick view
and the drawing model output in the tutorial in Chapters \ref{chap:tutorial_ideal} and \ref{chap:tutorial_real}, respectively.
Other tools are also available,
if they can be read in \netcdf file format, which is the output of \scaledg.

\begin{itemize}
\item Xarray / Matplotlib

\item GPhys / Ruby-DCL by GFD DENNOU Club
 \begin{itemize}
  \item URL: \url{http://ruby.gfd-dennou.org/products/gphys/}
  \item Note: \scalelib outputs the split files
  in {\netcdf} format according to domain decomposition by the MPI process.
  "gpview" and/or "gpvect" in {\gphys} can directly draw the split file without post-processing.
  \item How to install:
  On the GFD DENNOU Club webpage, the installation is explained for major OSs\\
  \url{http://ruby.gfd-dennou.org/tutorial/install/}\\
   \end{itemize}
\end{itemize}


% \subsubsection{Useful tools (not always necessary)}
% \begin{itemize}
%   \item {\bf Data conversion tool}: wgrib, wgrib2, NCL\\
%   These tools can generate input data readable by \scaledg.
%   In the tutorial for the real atmospheric experiment, wgrib is used.
%   \item {\bf Evaluation tool of computational performance}:The PAPI library\footnote{\url{http://icl.utk.edu/papi/}} is available.
% \end{itemize}
