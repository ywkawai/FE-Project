\section{Download source codes and Setting computational environment} \label{sec:download}
%====================================================================================

\subsubsection{Get source codes} %\label{subsec:get_source_code}
%-----------------------------------------------------------------------------------
The source codes for the latest release can be downloaded
from \url{https://ywkawai.github.io/FE-Project_web/download/}.\\
The directory \texttt{FE-Project-{\version}/} can be seen when the tarball file of the source code is extracted.
\begin{alltt}
 $ tar -zxvf FE-Project-{\version}.tar.gz
 $ ls ./FE-Project-{\version}/
\end{alltt}

\subsubsection{Set Makedef file and environmental variables} \label{subsec:environment}
%-----------------------------------------------------------------------------------

\scaledg is compiled using a Makedef file
specified in the environment variable ``\verb|SCALE_FE_SYS|.''
Several variations of the Makedef file are prepared in the directory \texttt{FE-Project-{\version}/sysdep/}.
According to your computational environment, 
choose a Makedef file in Table \ref{tab:makedef} and set the environment variable as
\begin{verbatim}
 $ export SCALE_FE_SYS="Linux64-gnu-ompi" (for example)
\end{verbatim}
If there is no Makedef file for your environment, 
create a Makedif file by modifying any existing one.
Because FE-Project uses SCALE library, 
you need to set a directory including SCALE library as 
\begin{verbatim}
 $ export SCALE="~/workspace/scale-5.X.X/" (for example)
\end{verbatim}

\begin{table}[htb]
\begin{center}
\caption{Examples of environments and their corresponding Makedef files.}
\begin{tabularx}{150mm}{|l|l|X|l|} \hline
 \rowcolor[gray]{0.9} OS/Computer & Compiler & MPI & Makedef file \\ \hline
                 & GNU & Open MPI & Makedef.Linux64-gnu-ompi \\ \cline{2-4}
 Linux OS x86-64 & Intel OneAPI   & Intel MPI & Makedef.Linux64-llvm-intel-impi \\ \hline
 Mac OS          & GNU            & Open MPI  & Makedef.MacOSX-gnu-ompi \\ \hline
 Fugaku          & Fujitsu        & Fujitsu MPI & Makedef.FUGAKU \\ \hline
\end{tabularx}
\label{tab:makedef}
\end{center}
\end{table}


All the environment variables applied at compile time are listed in Table \ref{tab:env_var_list}.
For example, if you would like to enable a thread parallelization with OpenMP, 
set a variable as
\begin{verbatim}
 $ export SCALE_ENABLE_OPENMP=T
\end{verbatim}

\begin{table}[htb]
\begin{center}
\caption{List of environment variables applied at compile time}
\begin{tabularx}{150mm}{|l|X|} \hline
 \rowcolor[gray]{0.9} Environment variable & Description \\ \hline
 SCALE                    & Path of SCALE library (required) \\ \hline
 SCALE\_FE\_SYS           & Selection of system (required) \\ \hline
 SCALE\_ENABLE\_OPENMP    & Enable to use OpenMP \\ \hline
%  SCALE\_ENABLE\_OPENACC   & Enable to use OpenACC \\ \hline
 SCALE\_DEBUG             & Use compile option for debug \\ \hline
 SCALE\_QUICKDEBUG        & Use compile option for quick debug (detect floating point error while maintaining speed-up option) \\ \hline
%  SCALE\_USE\_SINGLEFP     & Use single precision floating-point number (for all sources) \\ \hline
%  SCALE\_USE\_AGRESSIVEOPT & Conduct strong optimization (only for FX and intel, this can be a side effect) \\ \hline
%  SCALE\_DISABLE\_INTELVEC & Suppress option for vectorization (only with Intel compiler) \\ \hline
 SCALE\_NETCDF\_INCLUDE   & Include path of NetCDF library \\ \hline
 SCALE\_NETCDF\_LIBS      & Directory path of NetCDF library and specifying libraries \\ \hline
%  SCALE\_ENABLE\_PNETCDF   & Use Parallel NetCDF \\ \hline
%  SCALE\_COMPAT\_NETCDF3   & Limit to NetCDF3-compatible features \\ \hline
%  SCALE\_ENABLE\_MATHLIB   & Use numerical calculation library \\ \hline
 SCALE\_MATHLIB\_LIBS     & Directory path of numerical library and specifying libraries \\ \hline
%  SCALE\_ENABLE\_PAPI      & Use Performance Application Programming Interface (PAPI) \\ \hline
%  SCALE\_PAPI\_INCLUDE     & Include path of PAPI library \\ \hline
%  SCALE\_PAPI\_LIBS        & Directory path of PAPI library and specifying libraries \\ \hline
%  SCALE\_DISABLE\_LOCALBIN & Disable making local binary files in directory for test cases \\ \hline
%  SCALE\_IGNORE\_SRCDEP    & Ignore dependency check at the compile time \\ \hline
%  SCALE\_ENABLE\_SDM       & Use Super Droplet Method (SDM) model \\ \hline
\end{tabularx}
\label{tab:env_var_list}
\end{center}
\end{table}


For file I/O, \scaledg requires \netcdf.
In the most cases, the pathes of \netcdf may be automatically found by using ``nc-config'' command.
If the paths cannot be found automatically, 
you should set the environmental variables for \netcdf as follows:
\begin{verbatim}
 $ export SCALE_NETCDF_INCLUDE="-I/opt/netcdf/include"
 $ export SCALE_NETCDF_LIBS= \
        "-L/opt/hdf5/lib64 -L/opt/netcdf/lib64 -lnetcdff -lnetcdf -lhdf5_hl -lhdf5 -lm -lz"
\end{verbatim}


\section{Compile} \label{sec:compile}
%-----------------------------------------------------------------------------------

\subsubsection{Compile of \scaledg}

To build \scaledg and conduct a test case such as idealized mountain wave tests, 
move to a directory of the test case and execute a command as 
\begin{alltt}
 $ cd FE-Project-{\version}/model/atm_nonhydro3d/test/case/mountain_wave/linear_hydrostatic_case/
 $ make -j 4
\end{alltt}
The number of \verb|-j| option shows a number of parallel compile processes 
and specify this number according to your environment.
When a compilation is successful, 
the following three executable files are generated in the current directory.
\begin{verbatim}
 scale-dg  scale-dg_init  scale-dg_pp
\end{verbatim}

Note that we prepare various numerical experiments are prepared. 
The configuration files can be found in the directory of 
FE-Project-{\version}/model/atm\_nonhydro3d/test/case/*. 
It is useful for you to create a new experimental setting.

\subsubsection{Points to note}

\noindent If you want to compile them again, remove the created binary files by the following command:
\begin{verbatim}
 $ make clean
\end{verbatim}
Note that, this command does not remove the library already compiled.
When you recompile the files by changing the compilation environment and options, 
use the following command to remove all files created by compilation:
\begin{verbatim}
 $ make allclean
\end{verbatim}

