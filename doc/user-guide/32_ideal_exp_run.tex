\section{How to execute model} \label{sec:ideal_exp_run}
%====================================================================================


\subsubsection{Experimental setup}
This tutorial focuses on 
a test case of quasi two-dimensional mountain wave.
% In this idealized experiment, a cumulus appears and develops by providing a typical atmospheric vertical profile
% and an initial disturbance in the lower troposphere.
Table \ref{tab:setting_ideal} shows the experimental setting.

\begin{table}[htb]
\begin{minipage}{150mm}
\begin{center}
\caption{Experimental setting of idealized experiment}
\begin{tabularx}{150mm}{|l|X|X|} \hline
 \rowcolor[gray]{0.9} ~~ & Configuration & Note \\ \hline
 Number of MPI processes& 8: $x$-direction, 1: $y$-direction & 1 MPI parallelization \\ \hline
 Horizontal DOF & 40 elements $\times$ 8 DOF ($p=7$): $x$-direction, 1 element $\times$ 8 DOF ($p=7$): $y$-direction &  \\ \hline
 Vertical DOF & 8 elements $\times$ 8 DOF ($p=7$) & \\ \hline
 Horizontal element size & 2.5 km: $x$-direction, 2.5 km: $y$-direction & quasi 2D experiment in the $x$- and vertical directions. \\ \hline
 Lateral boundary condition & Periodic condition & Both eastward and northward \\ \hline
 Time step & 12 sec      &  \\ \hline
 Integration time     & 10 hour & The total number of time step is 3000.\\ \hline
 Time interval of data output & 1 hour  &  \\ \hline
%  Physical scheme & Only the microphysics scheme &
%  6-class single-moment bulk model \citep{tomita_2008} \\ \hline
%  Initial vertical profile & GCSS Case1 squall line \citep{Redelsperger2000}&
%  The wind profile based on \citet{Ooyama_2001} is given as the vertical shear. \\ \hline
%  Initial disturbance & Warm bubble & Horizontal radius of 4 km and
%  vertical radius of 3km with  maximum intensity of 3K.\\ \hline
\end{tabularx}
\label{tab:setting_ideal}
\end{center}
\end{minipage}
\end{table}

\subsubsection{Preparations} %\label{subsec:ideal_exp_prepare}
%------------------------------------------------------

As described in Section \ref{sec:compile},
move to a directory of target test case and compile \scaledg. 
\begin{verbatim}
  $ cd FE-Project-{\version}/model/atm_nonhydro3d/test/case/mountain_wave/linear_hydrostatic_case
  $ make -j4
\end{verbatim}
Then, 
the following three executable files are generated in the current directory.
\begin{verbatim}
 scale-dg  scale-dg_init  scale-dg_pp
\end{verbatim}

\subsubsection{Creating initial conditions} \label{subsec:ideal_exp_init}
%------------------------------------------------------

To create the initial conditions, the configuration file for \verb|scale-dg_init| is required.
The configuration file \verb|init_R20kmDX500m.conf| has been prepared
according to Table \ref{tab:setting_ideal}.
Reading the configuration file, \verb|scale-dg_init| calculates the stratified atmospheric structure and the initial disturbance.

The general form of the executable command in \scaledg is given as follows:
\begin{verbatim}
  $ mpirun -n [the number of processes] \\
    [executable binary name] [the configuration file]
\end{verbatim}
The number of processes using MPI parallel processing is given at [the number of processes]. 
The name of the executable binary is given to [executable binary name],  such as \verb|scale-dg|, \verb|scale-dg_init|, and so on.  
The configuration file, where the experimental settings are described, is given to [the configuration file].
%If \verb|sample/init_R20kmDX500m.conf| is used as the configuration file and 
If \verb|init.conf| is used as the configuration file and 
\verb|scale-dg_init| using eight-MPI parallel is executed, 
give the command as follows:
\begin{verbatim}
  $ mpirun  -n  8  ./scale-dg_init  ./init.conf
\end{verbatim}
\noindent
If it is successfully completed, 
the following message is output in the command line:

\msgbox{
 *** Start Launch System for SCALE-DG\\
 *** Execute preprocess? :  T\\
 *** Execute model?      :  F\\
 *** End   Launch System for SCALE-DG\\
}

Through the above, the following three files are generated under the given directory:
\begin{verbatim}
  init_LOG.pe000000
  init_00000101-000000.000.pe00000[0-7].nc
\end{verbatim}
where 
the number followed by \verb|pe| in the file name shows the process number of MPI.
In log file \verb|init_LOG.pe000000|,
detailed information that is not displayed in the command-line is recorded.
Although the eight MPI processes are used,
only the log file of the 0th MPI rank (master rank) is output as default.
If the execution is successfully finished, 
the statements below are output at the end of LOG file:

\msgbox{
 +++++ Closing LOG file\\
}

The file whose name ends with ``.nc''  is formatted by \netcdf.
It can be directly read by Xarray or GPhys/Ruby-DCL.


\subsubsection{Execution of simulation} %\label{subsec:ideal_exp_run}
%------------------------------------------------------

The number of parallel processes is required to be the same as that when creating the initial conditions.
The configuration file for the run is \verb|run.conf|. 

% \begin{verbatim}
%   $ cp  sample/run_R20kmDX500m.conf  ./run_R20kmDX500m.conf
%   $ mpirun  -n  2  ./SCALE-DG  run_R20kmDX500m.conf
% \end{verbatim}
\begin{verbatim}
  $ mpirun  -n  8  ./scale_dg  run.conf
\end{verbatim}

% If a computer satisfying the necessary requirements is used,
% the calculation is concluded within two minutes.
The following files are then generated under the given directory:
\begin{verbatim}
  LOG.pe000000
  history.pe00000[0-7].nc
\end{verbatim}
%In \verb|LOG.pe000000|, the detailed log that is not displayed in the command line is recorded.
When the execution finished normally,
the following message is output at the end of this LOG file:
\msgbox{
 +++++ Closing LOG file\\
}
The history files \verb|history.pe00000*.nc| contain the results of the calculation. 
