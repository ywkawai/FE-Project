\section{Setting integration period and time step} \label{sec:timeintiv}
%------------------------------------------------------

The integration period and time step are configured appropriately according to experimental design.
The time step depends on the spatial resolution of the model.
A shorter time step is sometimes required to avoid numerical instability.
The period of integration and the time step are configured in \namelist{PARAM_TIME} in \runconf.


\editboxtwo{
\verb|&PARAM_TIME| & \\
\verb| TIME_STARTDATE               = 2014, 8, 10, 0, 0, 0,| & Start date of integration: it is required for the calculation of the radiation process.\\
\verb| TIME_STARTMS                 = 0.D0,                | & Start date [mili sec]\\
\verb| TIME_DURATION                = 12.0D0,              | & Integration time [init is defined by \verb|TIME_DURATION_UNIT|]\\
\verb| TIME_DURATION_UNIT           = "HOUR",              | & Unit for \verb|TIME_DURATION|\\
\verb| TIME_DT                      = 60.0D0,              | & Time step for time integration\\
\verb| TIME_DT_UNIT                 = "SEC",               | & Unit for \verb|TIME_DT|\\
\verb|/|\\
}

\nmitem{TIME_DT} in \namelist{PARAM_TIME} is the time step for time integration, usually described as $\Delta t$.
For the time step with each component such as atmosphere and ocean component, 
we use \nmitem{TIME_DT} and \nmitem{TIME_DT_UNIT} in namelist for each component. 

