\section{Post-processing and visualization} \label{sec:ideal_exp_regrid}
%------------------------------------------------------
In this section, we explain post-processing and the method of drawing the calculation result.
In the tutorial, the distributed files in \netcdf format are merged into one file
and converted into a single \netcdf file that can be read by {\grads}.
The binary form makes it easy for users to analyze the result.
Link to the post-processing tool \regridTool compiled in Section \ref{sec:compile_post}:
\begin{verbatim}
  $ ln -s FE-Project-{\version}/bin/regrid_tool  ./
\end{verbatim}

The method of execution of \regridTool is the same as that of \scaledg, i.e.,
\begin{verbatim}
 $ mpirun -n [the number of the processes] ./regrid_tool [the configuration file]
\end{verbatim}
The configuration file \verb|regrid.conf| is intended for special uses of \regridTool.
Give this configuration file and execute it as follows:
% \begin{verbatim}
%   $ cp  sample/sno_R20kmDX500m.conf  ./sno_R20kmDX500m.conf
%   $ mpirun  -n  2  ./sno  ./sno_R20kmDX500m.conf
% \end{verbatim}
\begin{verbatim}
  $ mpirun  -n  8  ./regrid_tool  ./regrid_topo.conf
  $ mpirun  -n  8  ./regrid_tool  ./regrid.conf
\end{verbatim}
If there is no error message and the following message is displayed to the standard output,
the conversion is completed without problems:
\msgbox{
\verb|*** End   regrid_tool| \\
}

% The execution of \regridTool should be handled, 
% so that the number of MPI processes is a divisor of the number of the grids in the calculation domain which does not contain the halo region.
% The following file is generated under the same directory by this execution:
% \begin{alltt}
%   merged_history.pe000000.nc
% \end{alltt}
% The \netcdf file is the converted file obtained by merging the divided files,
% and readable in \grads by ``sdfopen'' command without the ``ctl'' file.

To confirm whether the flow calculation succeeded,
create a figure by the following command 
\begin{verbatim}
  $ bash visualize/visualize.sh
\end{verbatim}
This shell script call a Python script for visualizing simulation results. 
If it is successfully completed, 
the following files are generated in the directory of analysis/:

\begin{verbatim}
   U_t{0,7200,18000,36000}s.png
   W_t{0,7200,18000,36000}s.png
\end{verbatim}
Figure \ref{fig_ideal} shows the results of horizontal wind $U$ and vertical wind $W$ at $t=10$ hours.
If you obtain the same figures as Fig.\,\ref{fig_ideal}, 
post-processing is successfully finished.

\begin{figure}[htb]
\begin{center}
  \includegraphics[width=0.65\hsize]{./../figure/mountain_wave_U_t36000s.png}\\
  \includegraphics[width=0.65\hsize]{./../figure/mountain_wave_W_t36000s.png}\\
  \caption{The vertical section of horizontal and vertical winds after $t=10$ hours}
  \label{fig_ideal}
\end{center}
\end{figure}

