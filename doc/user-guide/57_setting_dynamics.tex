%------------------------------------------------------
This section describes how to configure regional and global dynamical cores.
For the parameters with mesh type and finite elements, 
we note that \namelist{PARAM_ATMOS} and \namelist{PARAM_ATMOS_DYN} are used, as described in Section \ref{sec:domain}.


\section{Temporal scheme}
\label{sec:atmos_dynamics_temporal_scheme}

The settings of temporal scheme are specified in \namelist{PARAM_ATMOS_DYN}. 
\editboxtwo{
 \verb|&PARAM_ATMOS_DYN  | & \\
 \verb|EQS_TYPE            = NONHYDRO3D_RHOT_HEVE,| & ; Choose from governing equations in Table \ref{tab:nml_atm_dyn_governeq}\\
 \verb|TINTEG_TYPE         = ERK_SSP_10s4o_2N,|     & ; Choose from temporal schemes in Table \ref{tb:HEVE_temporal_integ_choice} for HEVE \\
                                                    & ; and Table \ref{tb:HEVI_temporal_integ_choice} for HEVI \\
 \verb|TINTEG_TYPE_TRACER  = ERK_SSP3s3o,|    & ; Choose from temporal schemes in Table \ref{tb:HEVE_temporal_integ_choice} \\
 \verb|TIME_DT             = 5D-2,|           & ; Time step for calculation of dynamical process \\
 \verb|TIME_DT_UNIT        = sec,|            & ; Unit for \verb|TIME_DT| \\
 \verb|MODALFILTER_FLAG    = .false.,|         & ; Flag to set whether a modal filtering is used (see Section\ref{sec:modal_filter})\\
\verb|/             | & \\
}
%%
\nmitem{TIME_DT} is the time step for time integration with dynamical process.
It is also used as the time step for tracer advection.


\begin{table}[bth]
\begin{center}
  \caption{List of EQS\_TYPE available in dynamical process. 
  THERM means which variable is used in the thermodynamics equation. 
  $\rho \theta$ and $\rho e_t$ represent density-weighted potential temperature and total energy, respectively.}
  \label{tab:nml_atm_dyn_governeq}
  \begin{tabularx}{150mm}{lcXX} \hline
    \rowcolor[gray]{0.9}  EQS\_TYPE & THERM & Temporal strategy & Note \\ \hline
      \verb|NONHYDRO3D_RHOT_HEVE|  & $\rho \theta$ & HEVE & regional\\
      \verb|NONHYDRO3D_RHOT_HEVI|  & $\rho \theta$ & HEVI & regional \\
      \verb|NONHYDRO3D_ETOT_HEVE|  & $\rho e_t$ & HEVE & regional, experimental \\
      \verb|NONHYDRO3D_ETOT_HEVI|  & $\rho e_t$ & HEVI & regional, experimental \\
    \hline
      \verb|GLOBALNONHYDRO3D_RHOT_HEVE|  & $\rho \theta$ & HEVE & global \\
      \verb|GLOBALNONHYDRO3D_RHOT_HEVI|  & $\rho \theta$ & HEVI & global \\
      \verb|GLOBALNONHYDRO3D_ETOT_HEVE|  & $\rho e_t$ & HEVE & global, experimental \\
      \verb|GLOBALNONHYDRO3D_ETOT_HEVI|  & $\rho e_t$ & HEVI & global, experimental \\
    \hline
  \end{tabularx}
\end{center}
\end{table}

%t
\begin{table}[t]
\caption{Implicit and Explicit (IMEX) Runge--Kutta schemes available in HEVI temporal integration}
%% Explicit part
\begin{tabular}{l|cccc}
%\tophline
\hline
Abbrev. & Order & Num. of stages (I,E) & Note & Reference \\
\hline
IMEX\_ARK232 & 2 & (2,3) &  & \cite{Giraldo2013} \\
\hline
IMEX\_ARK324 & 3 & (3,4) &  & \cite{kennedy2003additive} \\
\hline
\end{tabular}
%%%%%
%\belowtable{} % Table Footnotes
\label{tb:HEVI_temporal_integ_choice}
\end{table}



\begin{table}[t]
\caption{Explicit Runge--Kutta schemes available in HEVE temporal integration}
%% Explicit part
\begin{tabular}{l|cccc}
%\tophline
\hline
Abbrev. & Order & Num. of stages & Note & Reference \\
\hline
ERK\_Euler & 1 & 1 & for debug & \\
\hline
ERK\_SSP\_2s2o & 2 & 2 & SSP & \cite{SHU1988439} \\
\hline
ERK\_SSP\_3s3o & 3 & 3 & SSP & \cite{SHU1988439} \\
ERK\_SSP\_4s3o & 3 & 4 & SSP & \\
ERK\_SSP\_5s3o\_2N2* & 3 & 5 & SSP & \cite{higueras2019new} \\
\hline
ERK\_RK4 & 4 & 4 & classical RK4 & \\
ERK\_SSP\_10s4o\_2N & 4 & 10 & SSP & \cite{Ketcheson2008RK4o10s} \\
\hline
\end{tabular}
%%%%%
%\belowtable{} % Table Footnotes
\label{tb:HEVE_temporal_integ_choice}
\end{table}

\section{Modal filtering} \label{sec:modal_filter}

For high-order DGM, 
numerical instability is likely to occur in advection-dominated flows 
because the numerical dissipations with the upwind numerical fluxes weaken. 
Furthermore, 
we adopted a collocation approach due to its computational efficiency. 
One drawback is that 
the aliasing errors with evaluations of the nonlinear terms can drive numerical instability. 

To suppress this numerical instability, 
a modal filter with an exponential function is used as an additional stabilization mechanism. 
The fundamental parameters are the order of $p_m$ and the non-dimensional decay strength $\alpha_m$. 
We applied the filter to the solution vector 
at the final stage of the RK scheme with a timestep $\Delta t$. 
Then, the decay timescale for the highest mode is approximately $\Delta t/\alpha_m$.

The switch of modal filtering exists in \namelist{PARAM_ATMOS_DYN} 
and 
the parameters are specified in \namelist{PARAM_ATMOS_DYN_MODALFILTER} as 
\editboxtwo{
 \verb|&PARAM_ATMOS_DYN  | & \\
 \verb|MODALFILTER_FLAG    = .true.,|  & ; Flag to set whether a modal filtering is used \\
\verb|/             | & \\
}
\editboxtwo{
 \verb|&PARAM_ATMOS_DYN_MODALFILTER  | & \\
 \verb|MF_ALPHA_h          = 1D-1,| & ; Order of modal filter in the horizontal direction  \\
 \verb|MF_ORDER_h          = 16,  | & ; Non-dimensional strength of modal filter in the horizontal direction \\
 \verb|MF_ALPHA_v          = 1D-1,| & ; Order of modal filter in the vertical direction  \\
 \verb|MF_ORDER_v          = 16,  | & ; Non-dimensional strength of modal filter in the vertical direction \\
\verb|/             | & \\
}

We set the order $p_m$, and decay coefficient $\alpha_m$ such that 
the strength of filter should ensure numerical stability while being as weak as possible. 

\section{Setting for boundary condition} \label{sec:atm_dynamics_bc}

Boundary conditions with atmospheric dynamical core are specified in \namelist{PARAM_ATMOS_DYN_BND}. 
For the default, slip and adiabatic conditions are imposed at rigid walls for velocity and heat, respectively. 
The boundary condition top and bottom boundaries are specified as 
\editboxtwo{
 \verb|&PARAM_ATMOS_DYN_BND  | & \\
  \verb|btm_vel_bc       = SLIP,| &; Velocity BC at bottom boundary \\
  \verb|top_vel_bc       = SLIP, | &; Velocity BC at top boundary \\
  \verb|btm_thermal_bc   = ADIABATIC,| &; Thermal BC at bottom boundary  \\
  \verb|top_thermal_bc   = ADIABATIC,|  &; Thermal BC at top boundary \\
\verb|/             | & \\
}


In the regional model, 
when lateral boundaries are not periodic,  
we should impose lateral boundary conditions as
\editboxtwo{
 \verb|&PARAM_ATMOS_DYN_BND  | & \\
  \verb|north_vel_bc       = SLIP,| &; Velocity BC at northern boundary \\
  \verb|south_vel_bc       = SLIP,| &; Velocity BC at southern boundary \\
  \verb|west_vel_bc       = SLIP,| &; Velocity BC at west boundary  \\
  \verb|east_vel_bc       = SLIP,| &; Velocity BC at east boundary \\
  \verb|north_thermal_bc   = ADIABATIC,| &; Thermal BC at northern boundary \\
  \verb|south_thermal_bc   = ADIABATIC,|  &; Thermal BC at southern boundary \\
  \verb|west_thermal_bc   = ADIABATIC,| &; Thermal BC at west boundary  \\
  \verb|east_thermal_bc   = ADIABATIC,|  &; Thermal BC at east boundary  \\
\verb|/             | & \\
}


\section{Setting for Coriolis force} \label{sec:coriolis}

\subsection{Regional model}

For regional model, 
the Coriolis parameter is zero as the default. 
Thus, you should set the parameters to introduce the Coriolis force in simulations.
There are two types of setting for the Coriolis parameter: $f$-/$\beta$-plane and sphere.
The type can be specified by \nmitem{CORIOLIS_type} in \namelist{PARAM_ATMOS_DYN_CORIOLIS}.

If \nmitem{CORIOLIS_type} is set to ``PLANE'', the Coriolis parameter $f$ is $f=f_0 + \beta (y-y_0)$.
The plane for $\beta=0$ is called $f$-plane, otherwise it is called $\beta$-plane.
The parameters of $f_0, \beta$ and $y_0$ is set with the parameters of \namelist{PARAM_ATMOS_DYN_CORIOLIS} as

\editboxtwo{
 \verb|&PARAM_ATMOS_DYN_CORIOLIS  | & \\
  \verb|CORIOLIS_type    = NONE,| &; Type of coriolis force: NONE, PLANE, SPHERE\\
  \verb|CORIOLIS_f0      = 0D0,| &; $f_0$  \\
  \verb|CORIOLIS_beta    = 0D0,| &; $\beta_0$  \\
  \verb|CORIOLIS_y0      = 0D0,| &; $y_0$ \\
\verb|/             | & \\
}

On the other hand, CORIOLIS\_type=SPHERE, 
the Coriolis parameter depends on the latitude as $f = 2\Omega \sin(\phi)$, 
where $\Omega$ and $\phi$ are angular velocity of the sphere and latitude, respectively.
The angular velocity of the sphere is set by \nmitem{CONST_OHM} parameter of \namelist{PARAM_CONST}.
The latitude of the individual grids is determined depending on the map projection.

\subsection{Global model}
For the global model, 
the Coriolis parameter is nonzero for the default and depends on the latitude. 
The angular velocity of rotation is set by \nmitem{CONST_OHM} parameter of \namelist{PARAM_CONST}.
If the Coriolis force is not considered, you should set \nmitem{CONST_OHM} to be zero. 


