
%==============================================================%
This user's manual is intended for users
of a regional and global atmospheric model \scaledg version \version.
The structure of this document is as follows:
Part \ref{part:overview} provides an overview of our model,
Part \ref{part:install} describes how to install \scaledg
Chapters \ref{chap:tutorial_ideal} and \ref{chap:tutorial_real}
explain the basic use of \scaledg
using examples of an idealized experiment and an realistic atmospheric experiment, respectively.
Since these chapters are constructed as a series of tutorials, 
it is recommended that beginning users of \scaledg read these chapters meticulously.
Parts \ref{part:basic_usel} and \ref{part:advance_use}
describe how to change the model configuration 
and explain data format and available functions and tools.
Since each section is closed itself basically, 
these chapter can be utilized as a dictionary.

% If you have any questions or comments, 
% please contact us through the users’ mailing list\\ $\langle \verb|scale-users@ml.riken.jp|\rangle$.


\section{What is FE-Project?} \label{subsec:scale_feature}
%--------------------------------------------------------------%

FE-project provides a library and sample programs for the discontinuous Galerkin methods
(DGM). Furthermore, we provide an atmospheric model with a regional/global dynamical
core based on DGM, SCALE-DG. In FE-Project, we use scalable Computing for Advanced
Library and Environment (SCALE) (\scaleweb), which is a basic software library of weather and
climate models of the earth and planets intended for widespread use.


Our project has the following advantages:
\begin{itemize}
\item 
FE-Project provides an open-source software under the `` MIT license''. It is free for use, modification, and redistribution, regardless of whether the user is a business or other enterprise.
\item 
FE-Project contains a regional and global atmospheric model called the \scaledg ( SCALE regional/global model with discontinuous Galerkin method).
\item 
In SCALE-DG, several physics processes are coupled to DG dynamical core using a framework for physical processes provided by SCALE library.
\end{itemize}
For the details of the license, the interested reader can refer to the file \texttt{FE-Project-\version/LICENSE} under the main directory. 
% This explanation of its use is also provided on the SCALE webpage (\scaleweb).
The general references of SCALE-DG are \cite{KT2023NumAccuracyDG,KT2025SCALEDG}. 
If SCALE-DG is used in your studies, 
please cite two our papers in addition to the reference papers of SCALE library \citep{nishizawa2015influence,sato2015impacts}. 


In this section, 
the concept of FE-Project and its relations to actual models are explained. 
It can be skipped, as it is not related directly to its practical use.

\clearpage
\Item{Relations between FE-Project (\scaledg) and SCALE}

\begin{figure}[htb]
\begin{center}
  \includegraphics[width=0.9\hsize]{./../figure/library.pdf}\\
  \caption{Aims of \scalelib (cite: Fig.\,1.1.1 of SCALE users guide)}
  \label{fig:scale}
\end{center}
\end{figure}
\begin{figure}[hbt]
\begin{center}
  \includegraphics[width=0.9\hsize]{./../figure/scale.pdf}\\
  \caption{Relationship between the library \scalelib and the model SCALE-RM (cite: Fig.\,1.1.2 of SCALE users guide)}
  \label{fig:scale-rm}
\end{center}
\end{figure}

\scalelib has been developed in RIKEN R-CCS with several outside contributors,
and its improvement and extension continue.
Figure \ref{fig:scale} shows the schematic concept of \scalelib.
As shown in this figure, SCALE aims to resolve various problems.
The development of \scalelib is considered in the context of its wide use
by devices ranging from small PC clusters to next-generation supercomputers.
For this purpose, scientists in meteorology/climate science
and computer science are cooperating. 
As shown in Fig. \ref{fig:scale-rm}, 
\scalelib manages the parallel processes,
file I/O, and inner-communication. 
\scalelib also provides the solver for atmospheric flow ( dynamical core based on FVM with Arakawa C-grid)
and physical processes such as micro-physics and radiation processes.
SCALE-RM is a regional atmospheric model that fully uses \scalelib.

%This has led to high computational performance of SCALE not only in supercomputers,
%such as the K Computer and the Fujitsu FX10,
%but also for general-purpose commercial computers,
%such as Intel processor-based machines.



Figure \ref{fig:FE-Project_relation} represents the relationship between the library \scalelib and FE-Project.
FElib provides Fortran modules for DGM which mange finite element, data communication, various temporal schemes, 
and many simple examples. 
In FElib, \scalelib library is used for file IO, error handling, and physical processes, etc.
\scaledg is constructed by combining functions provided by FElib.
\scaledg itself reads the input data of atmospheric status as prognostic variable,
and conducts time-integration.
Users can select a scheme in every component according to simulations they want.


\begin{figure}[hbt]
\begin{center}
  \includegraphics[width=0.9\hsize]{./../figure/FE-project_top.png}\\
  \caption{Relationship between the library \scalelib and FE-Project including \scaledg}
  \label{fig:FE-Project_relation}
\end{center}
\end{figure}


\section{Structure of \scaledg}  \label{subsec:sturcture_scale_rm}
%--------------------------------------------------------------%
Several physics in \scalelib library are available in \scaledg.
The components are categorized into three parts:
framework, dynamical core, and physical processes.
% Components with various schemes already implemented
% in the current version of \scaledg are listed below\footnote{Refer to \citet{scale_2015},\citet{satoy_2015b}, and \citet{nishizawa_2015} for the details of the model structure and the discretization method.}.

\subsubsection{Framework}
\begin{itemize}
 \item The three-dimensional (3D) Cartesian and cubed-sphere grid systems for regional and global models
 \item 2D domain decomposition by Message Passing Interface (MPI) communication
%  \item Several map projections commonly used
%  \item Domain nesting system ( one-way, i.e., data transfer from parent domain to child domain. )
%    \begin{itemize}
%     \item  On-line nesting: concurrent execution of multiple domains).
%     \item  Offline nesting: execution of computation in an inner domain after that in an outer domain.
  %  \end{itemize}
 \item Collective execution system of multiple cases, i.e., bulk job system
 \item \netcdf file I/O based on CF (Climate and Forecast) convention\footnote{\url{http://cfconventions.org/}}
  %  \begin{itemize}
  %  \item Selection of {\netcdf}3 and {\netcdf}4 formats
  %  \end{itemize}
 \item Generation of initial data for an ideal experiment
%  \item Generation of topographical and land-use data, converted from external data
%  \item Generation of initial and boundary data from external data
  %  \begin{itemize}
  %   \item 
%      Supporting inputs from the WRF-ARW\footnote{\url{http://www.wrf-model.org/}} and
  %     \grads \footnote{\url{http://cola.gmu.edu/grads/}} formats.
  %  \end{itemize}
\end{itemize}

\subsubsection{Dynamical core}
\begin{itemize}
 \item Governing equations: 3D fully compressible non-hydrostatic equations
 \item Spatial discretization: nodal discontinuous Galerkin method \citep{hesthaven2007nodal}
    \begin{itemize}
      \item Polynomial order associated with numerical accuracy can be arbitrarily chosen
      \item Numerical flux: Rusanov flux \citep{Rusanov1961}
      \item Stabilization mechanisms: Inherent numerical dissipation with numerical fluxes and explicit modal filtering
    \end{itemize}
 \item Temporal discretization: various type of Runge--Kutta (RK) schemes
    \begin{itemize}
      \item Horizontally and vertically explicit (HEVE) strategy with full explicit RK schemes
      \item Horizontally explicit and vertically implicit (HEVI) strategy with IMEX RK schemes
    \end{itemize}
 \item Guarantee of non-negative value for tracer advection:
    \begin{itemize}
      \item Preserving the non-negativity is ensured by a limiter \citep{LightPreserving2016}
    \end{itemize}
 \item Topography: Terrain-following coordinates
\end{itemize}


\subsubsection{Physical processes}
\begin{itemize}
\item Turbulence process: selectable from among the following
  \begin{itemize}
  \item \citet{smagorinsky1963general} \& \citet{lilly1962numerical}-type sub-grid scale turbulent model
    with the corrections by \citet{brown1994largeeddy} and \citet{Scotti_1993}
  % \item \citet{Deardorff_1980} sub-grid scale turbulent model
  % \item MYNN level 2.5 boundary scheme ( \citet{my_1982,nakanishi_2004,nakanishi_2009} )
  \end{itemize}
\item Cloud microphysics: selectable from among the following
  \begin{itemize}
  \item 3-class 1 moment bulk scheme \citep{kessler_1969}
  \item 6-class 1 moment bulk scheme \citep{tomita_2008}
% \item 6-class 2 moment bulk scheme \citep{sn_2014}
%  \item spectral bin scheme \citep{suzuki_etal_2010}
  \end{itemize}
% \item Radiation process: a k-distribution-based broadband radiation transfer model ( \citet{sekiguchi_2008} )
\item Surface models
  \begin{itemize}
  % \item Land model: heat diffusion/bucket model
  % \item Ocean model: selectable from among the following
  %   \begin{itemize}
  %   \item fixed to initial condition
  %   \item input from external data
  %   \item slab model
  %   \end{itemize}
  % \item Urban model: a single-layer canopy model \citep{kusaka_2001}
  % \item Heat transfer coefficient at surface: selectable from among the following
  %   \begin{itemize}
  %   \item The bulk method using the universal function \citep{beljaars_1991,wilson_2001,nishizawa_2018}
  %   \item Louis-type bulk method \citep{uno_1995}
  %   \end{itemize}
    \item Simple surface fluxes with a constant heat flux or constant bulk coefficients
  \end{itemize}
\end{itemize}
